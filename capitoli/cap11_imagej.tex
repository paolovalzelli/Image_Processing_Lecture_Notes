\chapter{\textcolor{colore8}{ImageJ laboratory}}

\hrulefill

\section{\textcolor{colore8}{ImageJ}}

\textbf{\textcolor{colore8}{ImageJ}} is a free multiplatform software, for which a huge variety of tools and forums are available.
To start working on an image it is sufficient to drop it on the toolbar, where a window shows general information about it (for example resolution, dimensions, type).

\subsection{\textcolor{colore8}{Basic operations}}

ImageJ allows to apply many algorithms of image processing, for example basic geometric (\texttt{Image $\longrightarrow$ Scale}, radiometric, mathematical and logical operations, both linear and non-linear.
To plot the histogram of the grey levels of the image it is possible to use \texttt{Analyze $\longrightarrow$ Histogram}, and by clicking on \texttt{List} the number of pixels with each grey level is shown in a list.
The interpolation policy can be managed in \texttt{Image $\longrightarrow$ Transform}.
Thresholding can be performed by clicking on \texttt{Image $\longrightarrow$ Adjust $\longrightarrow$ Threshold}.
In the same menu many other point operations are available.
\texttt{Process $\longrightarrow$ Math} provides many possible mathematical operations.
It is possible to carry out point by point operations between two images in \texttt{Process $\longrightarrow$ Image Calculator}.

It is possible to choose the fraction of outliers to be eliminated or to perform histogram equalisation in \texttt{Process $\longrightarrow$ Enhance Contrast}.

\subsection{\textcolor{colore8}{Local operations}}

The menu \texttt{Process} allows for many \textbf{local operations}, such as:

\begin{itemize}
    \item \texttt{Smooth} reduces noise;
    \item \texttt{Find Edges} detects edges;
    \item \texttt{Filters} provides a variety of filters, for example \texttt{Convolve} allows the choice of the kernel.
\end{itemize}

\texttt{Process $\longrightarrow$ Binary} provides a variety of logical operations.

\subsection{\textcolor{colore8}{Fourier operations}}

From the menu \texttt{Process $\longrightarrow$ FFT} (Fast Fourier transform) it is possible to change the settings of the transform operation in \texttt{FFT Options}.
\texttt{raw} gives as an output the spectrum in linear scale, instead of logarithmic, so that only the DC component is visible; whereas \texttt{complex} shows two windows, one for the real and one for the imaginary part of the spectrum.

\noindent The colour selector and the paintbrush in the toolbar allow for the manual modification of an image; therefore, they can be used to cut the undesired frequencies from a spectrum.

\noindent To apply a filter to the spectrum, it is possible to go to the menu \texttt{Process $\longrightarrow$ Filters $\longrightarrow$ Show circular masks} to choose the cutoff frequency to apply to the spectrum.

\noindent Once a mask is selected, it is possible to utilise it as a filter and carry out the filtering operation with a Fourier transform from \texttt{Image $\longrightarrow$ Duplicate}.
Another way to achieve this result is to use \texttt{Process $\longrightarrow$ FFT $\longrightarrow$ Custom Filter} or to manually cut out the undesired frequencies with a black paintbrush.

\section{\textcolor{colore8}{Image Analysis}}

It is possible to analyse only a portion of the image by selecting a line with the desired thickness using \texttt{Analyze $\longrightarrow$ Plot Profile} or by considering a wider region of interest.

\noindent To calculate the \textbf{signal to noise ratio}, it is necessary to select the mean and standard deviation in \texttt{Analyze $\longrightarrow$ Set Measurements} and then obtain the required information by choosing \texttt{Analyze $\longrightarrow$ Measure}.
The SNR can be used to assess the improvement in the quality of the image.

\noindent To compare two versions of the same image, it is possible to consider specific regions of interest (ROI) by selecting \texttt{Analyze $\longrightarrow$ Tools $\longrightarrow$ ROI Manager}.

\section{\textcolor{colore8}{Feature extraction}}

From \texttt{Analyze $\longrightarrow$ Analyze Particles}, it is possible to compute the count of the objects in the image, label them, and obtain a list of features such as area and grey level for each of them.
These features can be chosen from \texttt{Analyze $\longrightarrow$ Set Measurements}.
A threshold can be set on the size or on the circularity to exclude the undesired objects.

\noindent To perform a list of operations on a series of registered images in a set, it is possible to write a \textbf{Macro} code from \texttt{Plugins $\longrightarrow$ Macros $\longrightarrow$ Record}.
This way, every operation that is performed on the image is coded in a script that can later be repeated on another image.
To run a copy of this code, it is necessary to copy the macro in \texttt{Plugins $\longrightarrow$ New $\longrightarrow$ Macro}.
A complete list of the meanings of each operation in the macro is available on the ImageJ website.
Finally, to apply the same list of operations on a set of images, it is necessary to select \texttt{Process $\longrightarrow$ Batch $\longrightarrow$ Macro}, click on the desired folder and copy the macro code.

%\section{\textcolor{colore8}{}}

%\section{\textcolor{colore8}{}}