\chapter{Image compression}

\hrulefill

\section{Aim of image compression}

Since digital images can require a lot of storage, some \textbf{image compression} techniques are utilised to reduce storage space in archives or transmission time without changing the image too much.

\section{Redundancy and irrelevancy}

Some information can be eliminated from the image without reducing its usability.
Removing \textit{redundant} information does not affect the quality of the image, whereas removing \textit{irrelevant} information does.

\section{Lossless techniques}

\textbf{Lossless algorithms} allow the exact recovering of the original image, exploiting the fact that in an image some regions can be uniform or homogeneous. 
If noise causes small fluctuations on the grey level of consecutive pixels, it is possible to only store one grey level and the difference from that value for the other pixels.
Some techniques allow to store smaller numbers using fewer bits, thereby reducing the storing space required for the image.

\section{Lossy techniques}

It is also possible to cause small degradations to an image by applying \textbf{lossy algorithms} that can force approximately uniform regions in an image to become uniform.
Different levels of compression are available and can be chosen according to the application.