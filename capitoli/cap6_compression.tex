\chapter{Image Compression}
\label{ch:compression}

\section{Aim of Image Compression}

Since digital images can require substantial storage capacity, \textbf{image compression} techniques are utilised to reduce storage space in archives or transmission time without significantly altering the image.

\section{Redundancy and Irrelevancy}

Certain information can be eliminated from an image without compromising its usability.
Removing \textit{redundant} information does not affect image quality, whereas removing \textit{irrelevant} information does.

\section{Lossless Techniques}

\textbf{Lossless algorithms} allow for the exact recovery of the original image, exploiting the fact that some regions in an image may be uniform or homogeneous. 
If noise causes small fluctuations in the grey level of consecutive pixels, it is possible to store only one grey level and the difference from that value for subsequent pixels.
Certain techniques allow for the storage of smaller numbers using fewer bits, thereby reducing the space required for the image.

\section{Lossy Techniques}

It is also possible to introduce minor degradations to an image by applying \textbf{lossy algorithms}, which can force approximately uniform regions to become perfectly uniform.
Different compression levels are available and can be selected according to the specific application.