\color{black}

\chapter{Images and Visual Perception}
\label{ch:perception}

\section{Images in Physics}

Images constitute a fundamental data type with diverse applications in physics, ranging from computer vision and medical imaging techniques to cultural heritage analysis and satellite image processing.
In modern contexts, images are predominantly acquired and processed in digital format.

Physically, an image can be defined as a measurement of the radiant intensity emitted by a source and subsequently reflected by, or transmitted through, the observed subject.

\section{Radiometry}

Radiometry is the science of measuring electromagnetic radiation and its interaction with matter.
A radiation source is typically described by a spectrum of wavelengths, although monochromatic sources exist in specific applications.

The fundamental quantities are:

\begin{itemize}
    \item \textbf{radiant energy ($Q$):} The total energy carried by the radiation, measured in Joules [\unit{\joule}].
    \item \textbf{radiant flux ($\Phi$):} The energy transferred per unit time (power), measured in Watts [\unit{\watt}].
\end{itemize}

\subsection{Radiometric quantities}

Two critical quantities for image formation are:

\begin{itemize}
    \item \textbf{irradiance ($E$):} The radiant flux \textit{received} by a surface per unit area.
    \item \textbf{radiant exitance ($M$):} The radiant flux \textit{emitted} by a surface per unit area.
    
\end{itemize}
Consequently, irradiance is typically associated with the image plane (the sensor), while radiant exitance is associated with the object being imaged.

\begin{table}[htbp]
    \centering
    \caption{Fundamental radiometric quantities.}
    \label{tab:radiometric_quantities}
    \begin{tabular}{lccc}
        \toprule
        \textbf{Quantity} & \textbf{Symbol} & \textbf{Definition} & \textbf{SI Unit} \\
        \midrule
        Radiant Energy & $Q$ & - & \unit{\joule} \\
        Radiant Flux & $\Phi$ & $dQ/dt$ & \unit{\watt} \\
        Irradiance & $E$ & $d\Phi/dA$ & \unit{\watt\per\meter\squared} \\
        Radiant Exitance & $M$ & $d\Phi/dA$ & \unit{\watt\per\meter\squared} \\
        Radiant Intensity & $I$ & $d\Phi/d\omega$ & \unit{\watt\per\steradian} \\
        Radiance & $L$ & $\frac{d^2\Phi}{d\omega \, dA \cos\theta}$ & \unit{\watt\per\steradian\per\meter\squared} \\
        \bottomrule
    \end{tabular}
\end{table}

The technique that combines images of the same object obtained by exploring different regions of the electromagnetic spectrum to provide complementary information is known as \textbf{multispectral imaging}.

\subsection{Photometry}

Photometry is a subset of radiometry that deals specifically with visible light as perceived by the human eye. 
Photometric quantities are weighted by the spectral sensitivity of the eye.
The fundamental unit of \textbf{luminous intensity} is the \textit{candela} (\unit{\candela}), while the unit for \textbf{luminous flux}, which is the energy irradiated over unit time, is the \textit{lumen} (\unit{\lumen}).

\begin{table}[htbp]
    \centering
    \caption{Fundamental photometric quantities.}
    \label{tab:photometric_quantities}
    \begin{tabular}{lccc}
        \toprule
        \textbf{Quantity} & \textbf{Definition} & \textbf{SI Unit} & \textbf{Name} \\
        \midrule
        Luminous Flux & $F$ & - & \unit{\lumen} \\
        Illuminance & $dF/dA$ & \unit{\lumen\per\meter\squared} & \unit{\lux} \\
        Luminous Exitance (Emittance) & $dF/dA$ & \unit{\lumen\per\meter\squared} & \unit{\lux} \\
        Luminous Intensity & $dF/d\omega$ & \unit{\lumen\per\steradian} & \unit{\candela} \\
        Luminance & $\frac{d^2F}{d\omega \, dA \cos\theta}$ & \unit{\lumen\per\steradian\per\meter\squared} & \unit{\candela\per\meter\squared} ("nit") \\
        \bottomrule
    \end{tabular}
\end{table}

\section{Human Eye Response}

The human eye acts as an optical acquisition system. 
To quantify visual perception, we define a \textbf{luminosity function} (or spectral luminous efficiency function), which represents the eye's sensitivity to different wavelengths. 
This function acts as a transfer function, mapping the physical input (spectral radiance) to the perceptual output (brightness).

There are two distinct vision regimes:

\begin{itemize}
    \item \textbf{photopic vision:} Operates in well-lit conditions (daylight).
    \item \textbf{scotopic vision:} Operates in low-light conditions (night).
\end{itemize}

Both functions exhibit  peak efficiency at specific wavelengths.
Scotopic vision has a higher peak sensitivity and is shifted towards shorter wavelengths (blue) compared to photopic vision (the Purkinje effect).

\subsection{Intensity of a source}

The perceived intensity $L$ of a light source characterised by a spectral power distribution $P(\lambda)$ is determined by the integral of the power distribution weighted by the spectral sensitivity of the eye $V(\lambda)$:

\begin{equation}
    L = \int_0^\infty P(\lambda) \cdot V(\lambda) \, d\lambda
\end{equation}

\subsection{Anatomy of the human eye}

The eye consists of a lens system that focusses incident light onto a photosensitive layer called the \textbf{retina}. 
The retina converts electromagnetic radiation into neural signals, which are transmitted to the brain via the optic nerve. 
An adjustable diaphragm, the \textbf{iris}, regulates the amount of light entering the eye (pupil size) to adapt to varying illumination levels.

\begin{figure}[htbp]
    \centering
    \includegraphics[width=0.5\linewidth]{immagini/eye.png}
    \caption{Anatomy of the human eye.}
    \label{fig:eye}
\end{figure}

\subsection{Cones and rods}

The retina contains two primary types of photoreceptors: \textbf{rods} and \textbf{cones}.

Cones are densely concentrated in the \textbf{fovea} (the central region of the retina), providing high-acuity vision. 
Rod density peaks in the peripheral retina, making peripheral vision more sensitive to motion and low light. 
The region where the optic nerve exits the retina lacks photoreceptors, creating a \textbf{blind spot}.

There are three types of cones (S, M, L), roughly corresponding to sensitivities in the Blue, Green, and Red regions of the spectrum.

\begin{table}[htbp]
    \centering
    \caption{Comparison between rods and cones.}
    \label{tab:rods_cones}
    \begin{tabular}{lcc}
        \toprule
        \textbf{Feature} & \textbf{Rods} & \textbf{Cones} \\
        \midrule
        Sensitivity & High & Low \\
        Operating Range & Low light & High light ($> \SI{1}{\candela\per\meter\squared}$) \\
        Colour Vision & Monochromatic & Chromatic (Colour) \\
        Vision & Scotopic & Photopic \\
        \bottomrule
    \end{tabular}
\end{table}

\begin{figure}[htbp]
    \centering
    \includegraphics[width=0.7\linewidth]{immagini/spectral_sensitivity.png}
    \caption{Spectral sensitivity of the rods and cones.}
    \label{fig:spectral_sensitivity}
\end{figure}

\section{Colour Vision and Representation}

According to the \textbf{Young model} (\textcolor{gray}{\textbf{Young-Helmholtz trichromatic theory}}) (c. 1820), human colour vision relies on three distinct receptors. 
Consequently, any perceived colour can be described as a combination of three primary colours. 
A monochromatic wave stimulates the three cone types to varying degrees, producing a triplet of signals that the brain interprets as a specific colour.

A \textbf{colour space} is a mathematical model organising colours as tuples of numbers, typically three. 
The most common model is the \textbf{RGB} space, based on the \textit{Additive colour theory}.

In a classic \textit{Colour matching experiment}, an observer views a target colour and attempts to match it by adjusting the intensities of three primary light sources (Red, Green, Blue) projected onto a screen.
A generic colour $C$ can be matched by:

\begin{equation}
    C \equiv a_{\text{R}}(C) \cdot R + a_{\text{G}}(C) \cdot G + a_{\text{B}}(C) \cdot B
\end{equation}

where the coefficients $a$ represent the specific weights (intensities) of the primaries.

\begin{figure}[htbp]
    \centering
    \includegraphics[width=0.7\linewidth]{immagini/colour_matching.png}
    \caption{Colour matching experiment.}
    \label{fig:colour_matching}
\end{figure}

\textcolor{gray}{\textbf{Grassmann's laws}} of additive mixing state that if a colour $C_3$ is the sum of $C_1$ and $C_2$, then its components are the sum of the respective components:

\begin{align}
    C_3 \equiv & C_1 + C_2 \\ \equiv & 
    [a_{\text{R}}(C_1) + a_{\text{R}} (C_2)] \cdot R + 
    [a_{\text{G}}(C_1) + a_{\text{G}} (C_2)] \cdot G + 
    [a_{\text{B}}(C_1) + a_{\text{B}} (C_2)] \cdot B
\end{align}

Notably, not all spectral colours can be matched by adding positive amounts of RGB primaries. Some require a "negative" amount of a primary (which physically means adding that primary to the target side of the comparison to achieve a match):

\begin{equation}
    C + a_{\text{R}} \cdot R \equiv a_{\text{G}} \cdot G + a_{\text{B}} \cdot B \implies C \equiv -a_{\text{R}} \cdot R + a_{\text{G}} \cdot G + a_{\text{B}} \cdot B
\end{equation}

\subsection{RGB and CIE spaces}

\paragraph{RGB}
In the 3D RGB space, colours are vectors where components represent the intensities of the primaries. 
This system can be considered as a quantitative formulation of the additive law.
It is visualised as a unit cube:

\begin{itemize}
    \item \textbf{origin:} black $(0,0,0)$
    \item \textbf{opposite vertex:} white $(1,1,1)$
    \item \textbf{primaries:} red $(1,0,0)$, green $(0,1,0)$, blue $(0,0,1)$
    \item \textbf{antiprimaries \textcolor{gray}{(secondaries)} (CMY):} yellow $(1,1,0)$, cyan $(0,1,1)$, magenta $(1,0,1)$
\end{itemize}

\paragraph{CIE XYZ}
To avoid negative coefficients, the CIE (Commission Internationale de l'Éclairage) defined the XYZ colour space. 
Chromaticity is represented using normalised coordinates ($x, y$):

\begin{equation}
    x = \frac{R}{R + G + B}, \quad y = \frac{G}{R + G + B}, \quad z = \frac{B}{R + G + B}
\end{equation}

with the constraint $x + y + z = 1$. 
The $(x, y)$ plot (chromaticity diagram) in fig. \ref{fig:CIE} contains all perceptible colours.

The RGB components can be derived from the spectral power distribution $P(\lambda)$ and the colour matching functions $X$, $Y$, $Z$:

\begin{equation}
    R = \int_0^\infty P(\lambda) X (\lambda) \, d\lambda, \quad
    G = \int_0^\infty P(\lambda) Y (\lambda) \, d\lambda, \quad
    B = \int_0^\infty P(\lambda) Z (\lambda) \, d\lambda
\end{equation}

\begin{figure}[htbp]
    \centering
    \includegraphics[width=0.55\linewidth]{immagini/CIE.png} 
    \caption{The CIE chromaticity diagram.}
    \label{fig:CIE}
\end{figure}

\subsection{HSV (Hue, Saturation, Value)}

The HSV model describes colour in a manner more intuitive to human perception (and artistic mixing). 
It is represented as a hexagonal pyramid (cone):

\begin{itemize}
    \item \textbf{hue (angle):} Represents the colour type (red, yellow, green, cyan, blue, magenta).
    \item \textbf{saturation (radius):} Distance from the central axis (purity of the colour). White is at the centre (saturation 0).
    \item \textbf{value (or intensity or lightness) (height):} Brightness. Black is at the apex ($V=0$).
\end{itemize}

\subsection{CMY (Cyan, Magenta, Yellow)}

The CMY model is a \textbf{subtractive} model used primarily in printing, where pigments absorb (subtract) light. 
It is theoretically the complement of RGB:

\begin{equation}
    \begin{pmatrix} C \\ M \\ Y \end{pmatrix} = \begin{pmatrix} 1 \\ 1 \\ 1 \end{pmatrix} - \begin{pmatrix} R \\ G \\ B \end{pmatrix}
\end{equation}

\section{Human Visual System Characteristics}

The human visual system operates over a light intensity range of approximately 10 orders of magnitude, from the scotopic threshold to the glare limit. 
However, it cannot operate over this entire range simultaneously. 
Through a process called \textbf{brightness adaptation}, the eye adjusts its sensitivity to the current mean luminance level. 
Subjective brightness is generally a logarithmic function of light intensity \textcolor{gray}{(Weber-Fechner law)}.
The \textbf{Weber ratio} $\Delta I / I$ indicates the change in intensity $\Delta I$ that is discriminable by the observer with background brightness $I$.

\subsection{Mach band effect}

\textbf{Contrast} is the relative difference in brightness between regions. 
The Mach band effect demonstrates that the eye is not a simple photometer; it applies "edge enhancement". 
When a brightness ramp connects a dark and light region, the eye perceives an illusory overshoot (lighter band) and undershoot (darker band) at the edges, enhancing local contrast.

\begin{figure}[htbp]
    \centering
    \includegraphics[width=0.5\linewidth]{immagini/mach_band.png}
    \caption{Mach band effect.}
    \label{fig:mach_band}
\end{figure}

\subsection{Simultaneous contrast}

Perceived brightness is not absolute but context-dependent. 
A region of fixed intensity appears darker when placed against a bright background and lighter when placed against a dark background. 
This phenomenon, known as \textbf{simultaneous contrast}, highlights the visual system's reliance on relative comparison rather than absolute measurement.

\subsection{Colour blindness (daltonism)}

Colour blindness arises from a deficiency or absence of one or more cone types. 
It is often a genetic condition \textcolor{gray}{(X-linked recessive)}, predominantly affecting males and preventing the discrimination of certain colours (most commonly red-green).