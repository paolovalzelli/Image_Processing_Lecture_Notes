% --- Codifica e Font ---
\usepackage[utf8]{inputenc}
\usepackage[T1]{fontenc}
\usepackage{lmodern}         % Font vettoriale standard
\usepackage[english]{babel}  % Lingua inglese
\usepackage{microtype}       % Migliora la giustificazione del testo

% --- Geometria e Layout ---
\usepackage[a4paper, top=3cm, bottom=3cm, left=2.5cm, right=2.5cm, bindingoffset=8mm]{geometry}
\usepackage{emptypage}       % Rimuove header/footer dalle pagine vuote

% --- Header e Footer (Stile Libro) ---
\usepackage{fancyhdr}
\setlength{\headheight}{14.5pt}
\pagestyle{fancy}
\fancyhf{}
\fancyhead[LE,RO]{\thepage}
\fancyhead[LO]{\nouppercase{\rightmark}}
\fancyhead[RE]{\nouppercase{\leftmark}}
\renewcommand{\headrulewidth}{0.4pt}

% --- Matematica e Simboli ---
\usepackage{amsmath}
\usepackage{amssymb}
\usepackage{amsfonts}
\usepackage{mathtools}
\usepackage{siunitx}
\sisetup{mode=match}

% --- Macro Semantiche ---
\newcommand{\vect}[1]{\mathbf{#1}}
\newcommand{\matr}[1]{\mathbf{#1}}
\newcommand{\iu}{\mathrm{i}}
\newcommand{\fourier}[1]{\mathcal{F}\left\{#1\right\}}
\newcommand{\ifourier}[1]{\mathcal{F}^{-1}\left\{#1\right\}}
\DeclareMathOperator{\Reale}{Re}
\DeclareMathOperator{\Imag}{Im}

% --- Immagini e Grafica ---
\usepackage{graphicx}
\usepackage{float}
\usepackage{subcaption}
\usepackage[x11names]{xcolor}

% --- Configurazione Caption ---
\usepackage[font=small, labelfont=bf, tableposition=top, figureposition=bottom]{caption}

% --- Tabelle e Liste ---
\usepackage{booktabs}
\usepackage{multirow}
\usepackage{array}
\usepackage{tabularx}
\usepackage{enumitem}

%Definizione colonna centrata per tabularx (spostata qui per pulizia)
\newcolumntype{Y}{>{\centering\arraybackslash}X}

% --- Box Sostitutivo Leggero (No Tcolorbox) ---
% Definisco un box giallo semplice per le note importanti
\newsavebox{\impbox}
\newenvironment{importantbox}[1]
  {\par\vspace{1em}\noindent
   \begin{center}
   \begin{lrbox}{\impbox}
   \begin{minipage}{0.95\textwidth}
   \fcolorbox{yellow!50!black}{yellow!5!white}{%
     \begin{minipage}{0.95\linewidth}
       \medskip
       \textbf{#1}\par\smallskip
  }
  {
       \medskip
     \end{minipage}%
   }
   \end{minipage}
   \end{lrbox}
   \usebox{\impbox}
   \end{center}
   \par\vspace{1em}}

% --- Link e Riferimenti ---
\usepackage{hyperref}
\hypersetup{
    colorlinks=true,
    linkcolor=black,
    citecolor=black,
    filecolor=magenta,
    urlcolor=blue!80!black,
    pdftitle={Image Processing Lecture Notes},
    pdfauthor={Paolo Valzelli},
    bookmarksnumbered=true
}