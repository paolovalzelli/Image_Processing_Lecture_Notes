\chapter{Images and visual perception}

\hrulefill

\section{Images in physics}

Images are a type of data with many applications in physics, such as computer vision, medical techniques, cultural heritage, and satellite image processing.
Nowadays, images are always in digital format.

An image is a measure of the intensity of the radiation that is emitted from a source and later reflected by or transmitted through the observed subject.

\section{Radiometry}

Radiometry measures radiation and its interaction with matter.
The source of radiation is often described by a spectrum of different wavelengths, but in some cases, there can also be monochromatic sources.
The amount of energy carried by radiation is called \textbf{radiant energy} (E), while the power of radiation, that is, the energy per unit time, is called \textbf{radiant flux} (Q).

\subsection{Radiometric quantities}

Usually, \textbf{irradiance} (E) indicates the \textit{received} amount of radiant flux per unit area, while \textbf{radiant exitance} (L) indicates the \textit{emitted} amount of radiant flux per unit area.
Therefore, the former is associated with the image and the latter with the object.

\begin{table}[htb]
    \centering
    \begin{tabular}{||c|c|c||}
    \hline
    \textbf{Quantity} & \textbf{Definition} & \textbf{Unit} \\
    \hline
    Irradiance & $dQ/dA$ & $\left[ W/m2 \right]$ \\
    \hline
    Radiant exitance & $dQ/dA$ & $\left[ W/m2 \right]$ \\
    \hline
    Radiant intensity & $dQ/d\omega$ & $\left[ W/sterad \right]$ \\
    \hline
    Radiance & $dQ/d\omega dA$ & $\left[ W/sterad \cdot m^2 \right]$ \\
    \hline
    \end{tabular}
    \caption{Radiometric quantities.}
    \label{tab:radiometric_quantities}
\end{table}

The technique that combines different images of the same object, obtained by exploring different parts of the electromagnetic spectrum, to provide complementary information, is called \textbf{multispectral imaging}.

\subsection{Photometry}

Photometry is the field of radiometry that only studies visible light.
Photometric quantities resemble radiometric quantities, but have distinct, more specific names.
The fundamental unit of \textbf{luminous intensity} is the \textit{candela} (cd), while the unit for the \textbf{luminous flux} (F) is the \textit{lumen}.

\begin{table}[htb]
    \centering
    \begin{tabular}{||c|c|c||}
    \hline
    \textbf{Quantity} & \textbf{Definition} & \textbf{Unit} \\
    \hline
    Illuminance & $dF/dA$ & $\left[ lumen/m^2 = lux\right]$ \\
    \hline
    Luminous exitance (emittance) & $dF/dA$ & $\left[ lumen/m^2 = lux\right]$ \\
    \hline
    Intensity & $dF/d\omega$ & $\left[ lumen/sterad = candela\right]$ \\
    \hline
    Luminance & $dF/d\omega dA$ & $\left[ lumen/sterad \cdot m^2 \right]$ \\
    \hline
    \end{tabular}
    \caption{Photometric quantities.}
    \label{tab:radiometric_quantities}
\end{table}

\section{Human eye response}

The human eye can be considered an acquisition system that provides an image.
It is necessary to define a \textbf{luminosity function} (that is a response function) that measures the output (that is, the image, or the measure of the intensity of the radiation) as a function of the input (the actual intensity).
This can be considered a spectral sensitivity: the efficiency of the eye for a range of wavelengths.

There are two distinct luminosity functions:

\begin{itemize}
    \item \textbf{photopic vision}, for everyday light;
    \item \textbf{scotopic vision}, for low light level.
\end{itemize}

In both cases, the efficiency is not maximum at every wavelength but shows a peak (which is typical in a real device).
Scotopic vision shows a higher maximum efficiency and is shifted towards shorter wavelengths (blue).

\subsection{Intensity of a source}

The intensity of a light source ($L$) characterized by a spectral power distribution ($P(\lambda)$) depends on the spectral sensitivity of the human eye ($V(\lambda)$):

\begin{equation}
    L = \int_0^\infty P(\lambda) \cdot V(\lambda) d \lambda
\end{equation}

\subsection{Anatomy of the human eye}

A lens makes the light converge to (almost) a single point on the light sensitive layer, called the retina, that converts the incoming radiation to an electric signal that is sent to the brain by the optical nerve.
A diaphragm called the iris regulates the amount of light that reaches the acquisition device.

\subsection{Cones and rods}

There are two types of photoreceptors inside the retina: \textbf{rods} and \textbf{cones}.

\begin{table}[htb]
    \centering
    \begin{tabular}{||c|c||}
    \hline
    \textbf{Rods} & \textbf{Cones} \\
    \hline
    High sensitivity & Low sensitivity \\
    \hline
    Low light & High light ($> 1 cd/m^2$) \\
    \hline
    Monochromatic & Colour \\
    \hline
    Scotopic vision & Photopic vision \\
    \hline
    \end{tabular}
    \caption{Caption}
    \label{tab:rods_cones}
\end{table}

The highest concentration of cones occurs in the fovea, while the concentration of rods peaks further away, making the peripheral vision more sensitive in low light conditions.
A blind spot appears due to the presence of the optic nerve.

Rods and cones have different spectral sensitivities.
There are three kinds of cones, characterized by three different spectral sensitivities peaked around the colours red, green, and blue.

\section{Colour vision and representation}

In the \textbf{Young model} (1820) every colour can be described as a combination of three primary colours, so that, providing one monochromatic wave to the eye, a triplet of values, that correspond to the efficiencies of the three kinds of cones, could be used to decode and represent it.
Each monochromatic wave is mapped into a triplet of signals that the human brain perceives as a distinct colour.

A \textbf{colour space} is a model describing the way colours can be represented as tuples of numbers, typically as three values.
The most common colour space is \textbf{RGB}, a quantitative formulation of the \textit{Additive law} (also known as \textit{Tristimulus theory}).

In the \textit{Colour matching experiment}, an observer compares a screen on which a desired colour is projected and one where there is a combination of red, green, and blue light, adjusting the quantities so that they match.
This is one way to manually map every possible colour.

A colour $C$ can be reproduced as a combination of the three primary colours:

\begin{equation}
    C \equiv a_R (C) \cdot R + a_G (C) \cdot G + a_B (C) \cdot B
\end{equation}

where the coefficients $a$ represent the fraction of the primary colours.

The \textit{additive property} states that if a colour can be achieved as the sum of two other colours, it is:

\begin{equation}
    C_3 \equiv C_1 + C_2  \equiv [a_R (C_1) + a_R (C_2) ] \cdot R + [a_G (C_1) + a_G (C_2) ] \cdot G + [a_B (C_1) + a_B (C_2) ] \cdot B
\end{equation}

The colour matching functions sometimes require a negative value for some spectral colours in the blue-green region ($C_{vb}$), so they cannot be described by the tristimulus theory.
These colours can be matched by adding one of the primaries to the colour itself:

\begin{equation}
    C + a_R (C) \cdot R \equiv a_G (C) \cdot G + a_B (C) \cdot B \text{ } \Longrightarrow \text{ } C \equiv -a_R (C) \cdot R + a_G (C) \cdot G + a_B (C) \cdot B
\end{equation}

\subsection{RGB and CIE}

In the RGB (3D) space, each colour can be represented as a vector whose components make up the intensities of the three primaries.
This space can be visualized as a cube whose vertices represent the colours:

\begin{itemize}
    \item White (0,0,0);
    \item Red (1,0,0), primary;
    \item Green (0,1,0), primary;
    \item Blue (0,0,1), primary;
    \item Yellow (1,1,0), antiprimary;
    \item Cyan (0,1,1), antiprimary;
    \item Magenta (1,0,1), antiprimary;
    \item Black (1,1,1).
\end{itemize}

In the \textbf{CIE} (Commission Internationale d'Eclairage) the chromaticity of a colour is represented with three parameters (\textit{fractional} primaries):

\begin{equation}
    x = \frac{R}{R + G + B}; y = \frac{G}{R + G + B}; z = \frac{B}{R + G + B} \text{ } ; \text{ } x +y +z = 1
\end{equation}

For a given luminance, the chromaticity of a colour can be specified by the two parameters $x$ and $y$.
This system can represent even the colours that cannot be described by the RGB space.

\begin{figure}
    \centering
    \includegraphics[width=0.7\linewidth]{immagini/CIE.png}
    \caption{CIE colour space.}
    \label{fig:CIE}
\end{figure}

The formulas for converting the CIE coordinates to the RGB systems are given by:

\begin{equation}
    R = \int_0^\infty P(\lambda) \cdot X(\lambda) d \lambda \text{ } ; \text{ } 
    G = \int_0^\infty P(\lambda) \cdot Y(\lambda) d \lambda \text{ } ; \text{ } 
    B = \int_0^\infty P(\lambda) \cdot Z(\lambda) d \lambda
\end{equation}

where $P(\lambda)$ is the spectral distribution of the light to be represented, and $X,Y,Z$ are the CIE primaries.

\subsection{HSV}

The \textbf{HSV} (Hue, Saturation, Value or Intensity or Lightness) colour space  can be represented as a hexagonal pyramid that describes colours in a way that is more similar to an artist's approach.
The hue represents the colour as an angle, so that in the corners there are red, yellow, green, cyan, blue and magenta.
Saturation increases further away from the centre, where white is located, and value increases going up, towards the base, so that in the vertex there is black.

\subsection{CMY}

The \textbf{CMY} (Cyan, Magenta, Yellow) space is based on a subtractive model, which is useful for colour printing.
The coordinates of a point in this space are complementary to those of the corresponding point in the RGB space:

\begin{equation}
\begin{pmatrix}
    C \\
    M \\
    Y
\end{pmatrix}
=
\begin{pmatrix}
    1 \\
    1 \\
    1
\end{pmatrix}
-
\begin{pmatrix}
    R \\
    G \\
    B
\end{pmatrix}
\end{equation}

\section{Human visual system}

From the scotopic threshold to the glare limit, the range of light intensity covers approximately ten orders of magnitude.
The subjective brightness is a logarithmic function of the light intensity and the visual system changes its overall sensitivity through \textit{brightness adaptation}, that requires some time, so that it cannot operate simultaneously over the entire range.

\subsection{Mach band effect}

\textbf{Contrast} is the difference in brightness between two regions in an image.
The human eye tends to increase the local, perceived contrast if there are step edges in brightness, so that a uniform colour band is perceived as not uniform.

\subsection{Simultaneous contrast}

the perceived brightness of a region also depends on the background intensity, so that equally bright regions might appear darker or lighter due to the surrounding background.
Phantom spots are another example of the contrast that the human eye adds to an image.

\subsection{Daltonism}

If a kind of the cones is affected by a problem, some colours cannot be distinguished by the human eye.
This condition is a genetic disease that predominantly occurs in men.